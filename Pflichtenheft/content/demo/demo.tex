
\newpage

\paragraph{\LARGE Struktur eines Beispielprotokolls}

\section{Aufgabe 1}
	\subsection{Vorbereitung}
	Was für Vorbereitungen waren nötig, um die Aufgabe zu lösen. Hier können beispielsweise Konfigurationen von Tools oder VM's beschrieben werden.
	\subsection{Durchführung}
	Wie ist die konkrete Durchführung vonstattengegangen. Wichtig hierbei ist die detaillierte Beschreibung des ``Versuches'' eventuell auch unter Zuhilfenahme von Diagrammen oder Listings.
	\subsection{Fazit}
	Hat der Versuch geklappt? Welche Probleme traten auf? Welche Ergebnisse wurden erzielt. Wie lassen sich die Ergebnisse interpretieren.
	
\begin{quote}
\textit{Hinweis: Wichtig ist, dass bei der Dokumentation der Ergebnisse sämtliche Quellen, die bei der Erfüllung der Aufgabe hilfreich waren, korrekt zitiert werden. (Stackoverflow ist keine (alleinige) Quelle und sollte bei Bedarf durch weitere Literatur untermauert werden - Warum funktioniert die Lösung aus einem Forum etc.).}
\end{quote}
\section{Aufgabe 2}
	\subsection{Vorbereitung}
	...
	\subsection{Durchführung}
	...
	\subsection{Fazit}
	...
\newpage
\paragraph{{\LARGE Demonstration eines Dokumentes (nach Matthias Pospiech)}}

% ============================================================
\section{Textauszeichnungen}

Die Standard Text Schalter erzeugen
\textbf{fett},
\textit{kursiv\footnote{kursiv und schräggestellt sind unterschiedliche Schriften und daher nicht das gleiche !} (italic)},
\textsl{schräggestellt (slanted)},
\textsf{serifenlos (grotesk)},
\textsc{Kapitälchen} und
\texttt{Schreibmaschinenschrift}.
Sowie beliebige Kombinationen derselben:
\textit{\textbf{fett kursiv}},
\textsl{\textbf{fett schräg}},
\textsf{\textbf{fett serifenlos}},
\textsc{\textbf{Fette Kapitälchen}}
und
\textsl{\textsf{serifenlos schräg}}. Je nach Schrift sind jedoch nicht alle Kombinationen möglich. In dem Fall bekommt man die Fehlermeldung `Some font shapes were not available, defaults substituted.'

% ============================================================

\section{Text und Zitate}
% ------------------------------------------------------------
\subsection{Linksbündig}
%
Mit dem Paket \package{ragged2e} und der Umgebung \env{FlushLeft} bzw. dem Befehl \command{Raggedright} wird Flattersatz (Gegenteil von Blocksatz) gegenüber dem \LaTeX{} Standard noch verbessert.

\begin{FlushLeft}
Aliquam ultrices libero hendrerit diam. Vestibulum ultrices sapien sit amet elit. Quisque tempor nisl eu sem. Nam lorem lectus, viverra nec, rutrum quis, lobortis nec, magna. Praesent hendrerit tortor vitae elit. Vivamus sed leo at mi elementum semper. Lorem ipsum dolor sit amet, consectetuer adipiscing elit. Aliquam eu nisi. Nam eget dui a tortor congue imperdiet. Etiam mattis. Nam tristique. Sed malesuada neque ut leo. Aenean est. In id augue.
\end{FlushLeft}
% ------------------------------------------------------------
\subsection{Rechtsbündig}
%
Mit dem Paket \package{ragged2e} und der Umgebung \env{FlushRight} bzw. dem Befehl \command{RaggedLeft} wird Flattersatz (Gegenteil von Blocksatz) gegenüber dem \LaTeX{} Standard noch verbessert.

\begin{FlushRight}
Aliquam ultrices libero hendrerit diam. Vestibulum ultrices sapien sit amet elit. Quisque tempor nisl eu sem. Nam lorem lectus, viverra nec, rutrum quis, lobortis nec, magna. Praesent hendrerit tortor vitae elit. Vivamus sed leo at mi elementum semper. Lorem ipsum dolor sit amet, consectetuer adipiscing elit. Aliquam eu nisi. Nam eget dui a tortor congue imperdiet. Etiam mattis. Nam tristique. Sed malesuada neque ut leo. Aenean est. In id augue.
\end{FlushRight}
%
% ------------------------------------------------------------
\subsection{Zentierter Text}
%
Zentierter Text, erstellt mit \env{center}\footnote{nicht verwenden innerhalb \env{table} oder \env{figure}!}.
\begin{center}
Aliquam ultrices libero hendrerit diam. Vestibulum ultrices sapien sit amet elit. Quisque tempor nisl eu sem. Nam lorem lectus, viverra nec, rutrum quis, lobortis nec, magna. Praesent hendrerit tortor vitae elit. Vivamus sed leo at mi elementum semper. Lorem ipsum dolor sit amet, consectetuer adipiscing elit. Aliquam eu nisi. Nam eget dui a tortor congue imperdiet. Etiam mattis. Nam tristique. Sed malesuada neque ut leo. Aenean est. In id augue.
\end{center}

% ------------------------------------------------------------
\subsection{Zitat mit quote}
Ein Text in einer \env{quote} Umgebung\footnote{Für Zitate eignet sich allerdings eher die Umgebung \env{blockquote} vom Paket \package{csquotes}}. Der gesammte Text innerhalb der Umgebung wird von beiden Seiten eingerückt.

\begin{quote}
Aliquam ultrices libero hendrerit diam. Vestibulum ultrices sapien sit amet elit. Quisque tempor nisl eu sem. Nam lorem lectus, viverra nec, rutrum quis, lobortis nec, magna. Praesent hendrerit tortor vitae elit. Vivamus sed leo at mi elementum semper. Lorem ipsum dolor sit amet, consectetuer adipiscing elit. Aliquam eu nisi. Nam eget dui a tortor congue imperdiet. Etiam mattis. Nam tristique. Sed malesuada neque ut leo. Aenean est. In id augue.
\end{quote}



% ============================================================
\section{Listen}

% ------------------------------------------------------------
\subsection{itemize}
Dies ist die Standard Aufzählungsliste von \LaTeXe. Sie hat einen Abstand zwischen den Einträgen um bei Längeren Zeilen das Lesen zu erleichtern.

\begin{itemize}
   \item Lorem ipsum dolor sit amet, consectetur adipisicing elit, sed do eiusmod tempor incididunt ut labore et dolore magna aliqua.
%
   \item Lorem ipsum dolor sit amet, consectetur adipisicing elit, sed do eiusmod tempor incididunt ut labore et dolore magna aliqua.
%
   \item Lorem ipsum dolor sit amet, consectetur adipisicing elit, sed do eiusmod tempor incididunt ut labore et dolore magna aliqua.
\end{itemize}

mit mehr als einer Ebene

\begin{itemize}
   \item Lorem ipsum dolor sit amet, consectetur adipisicing elit, sed do eiusmod tempor incididunt ut labore et dolore magna aliqua.
%
   \begin{itemize}
      \item Lorem ipsum dolor sit amet, consectetur adipisicing elit, sed do eiusmod tempor incididunt ut labore et dolore magna aliqua.
      %
      \begin{itemize}
         \item Lorem ipsum dolor sit amet, consectetur adipisicing elit, sed do eiusmod tempor incididunt ut labore et dolore magna aliqua.
         %
         \item Lorem ipsum dolor sit amet, consectetur adipisicing elit, sed do eiusmod tempor incididunt ut labore et dolore magna aliqua.
         %
         \item Lorem ipsum dolor sit amet, consectetur adipisicing elit, sed do eiusmod tempor incididunt ut labore et dolore magna aliqua.
      \end{itemize}
      %
      \item Lorem ipsum dolor sit amet, consectetur adipisicing elit, sed do eiusmod tempor incididunt ut labore et dolore magna aliqua.
      %
      \item Lorem ipsum dolor sit amet, consectetur adipisicing elit, sed do eiusmod tempor incididunt ut labore et dolore magna aliqua.
   \end{itemize}
\end{itemize}

% ------------------------------------------------------------
\subsection{enumerate}
Dies ist die Standard Nummerierungsliste von \LaTeXe{}. Sie hat einen Abstand zwischen den Einträgen um bei Längeren Zeilen das Lesen zu erleichtern.

\begin{enumerate}
   \item Lorem ipsum dolor sit amet, consectetur adipisicing elit, sed do eiusmod tempor incididunt ut labore et dolore magna aliqua.
%
   \begin{enumerate}
      \item Lorem ipsum dolor sit amet, consectetur adipisicing elit, sed do eiusmod tempor incididunt ut labore et dolore magna aliqua.
      %
      \begin{enumerate}
         \item Lorem ipsum dolor sit amet, consectetur adipisicing elit, sed do eiusmod tempor incididunt ut labore et dolore magna aliqua.
         %
         \item Lorem ipsum dolor sit amet, consectetur adipisicing elit, sed do eiusmod tempor incididunt ut labore et dolore magna aliqua.
         %
         \item Lorem ipsum dolor sit amet, consectetur adipisicing elit, sed do eiusmod tempor incididunt ut labore et dolore magna aliqua.
      \end{enumerate}
      %
      \item Lorem ipsum dolor sit amet, consectetur adipisicing elit, sed do eiusmod tempor incididunt ut labore et dolore magna aliqua.
      %
      \item Lorem ipsum dolor sit amet, consectetur adipisicing elit, sed do eiusmod tempor incididunt ut labore et dolore magna aliqua.
   \end{enumerate}
\end{enumerate}

% ------------------------------------------------------------
%
\subsection{Kompakte Liste}

Die eleganteste Lösung bietet das Paket \package{enumitem} mit der Option `noitemsep'
\begin{itemize}[noitemsep]
\item Diese Umgebung
\item sollte man nur nutzen,
\item wenn die Einträge nicht länger als eine Zeile sind
\end{itemize}

% ------------------------------------------------------------

\subsection{Beliebige Labels}

Dies ist die \env{enumerate} Umgebung des Paketes \package{enumitem} mit der Option [label=\alph{enumi})].

\begin{enumerate}[label=\alph{enumi})]
         \item Lorem ipsum dolor sit amet, consectetur adipisicing elit, sed do eiusmod tempor incididunt ut labore et dolore magna aliqua.
         %
         \item Lorem ipsum dolor sit amet, consectetur adipisicing elit, sed do eiusmod tempor incididunt ut labore et dolore magna aliqua.
\end{enumerate}

% ------------------------------------------------------------
\subsection{description}
%
\begin{description}
\item[Beschreibung] Die Umgebung \env{description} dient zum Erläutern.
\item[Flüsse] Elbe, Weser
\item[Meere] Indischer, Pazifischer Ozean, Mittelmeer
\end{description}

% ============================================================
%\clearpage
\section{Abbildungen}

% ------------------------------------------------------------

\subsection{Normale figure Umgebung}
Nulla malesuada porttitor diam. Donec felis
erat, congue non, volutpat at, tincidunt tristique, libero. Vivamus
viverra fermentum felis. Donec nonummy pellentesque ante. Phasellus
adipiscing semper elit. Proin fermentum massa ac quam. Sed diam
turpis, molestie vitae, placerat a, molestie nec, leo. Maecenas
lacinia.

\begin{figure}[H]
   \centering
   \includegraphics[width=0.3\textwidth]{images/latex}
   \caption[Normale Figure Umgebung]{Dies ist eine lange Abbildungungsbeschriftung. Dies ist eine lange Abbildungungsbeschriftung. Dies ist eine lange Abbildungungsbeschriftung.}
\end{figure}

\subsection{Normale figure Umgebung, zwei Bilder nebeneinander}
\begin{figure}[H]
   \centering
   \begin{minipage}{0.4\linewidth}
   \includegraphics[width=1.0\linewidth]{images/latex}
   \caption[Normale Figure Umgebung]{Dies ist eine lange Abbildungungsbeschriftung. Dies ist eine lange Abbildungungsbeschriftung. Dies ist eine lange Abbildungungsbeschriftung.}
	\end{minipage}\hspace{2em}%\hfill%
   \begin{minipage}{0.4\linewidth}
   \includegraphics[width=1.0\linewidth]{images/latex}
   \caption[Normale Figure Umgebung]{Dies ist eine lange Abbildungungsbeschriftung. Dies ist eine lange Abbildungungsbeschriftung. Dies ist eine lange Abbildungungsbeschriftung.}
	\end{minipage}
\end{figure}

%

% ------------------------------------------------------------
%
\subsection{subcaption in minipages}
	Nam ipsum ligula, eleifend at, accumsan nec, suscipit a,
	ipsum. Morbi blandit ligula feugiat magna. Nunc eleifend consequat
	lorem. Sed lacinia nulla vitae enim. Pellentesque tincidunt purus
	vel magna. Integer non enim. Praesent euismod nunc eu purus. Donec
	bibendum quam in tellus. Nullam cursus pulvinar lectus. Donec et mi.
	Nam vulputate metus eu enim. Vestibulum pellentesque felis eu
	massa.
	
	\begin{figure}[H]
	\begin{minipage}[b]{.5\linewidth}
		\centering
		\includegraphics[width=0.8\linewidth]{images/latex}
		\subcaption{A subfigure}\label{fig:1a}
	\end{minipage}%
	\begin{minipage}[b]{.5\linewidth}
		\centering
		\includegraphics[width=0.8\linewidth]{images/latex}
		\subcaption{Another subfigure}\label{fig:1b}
	\end{minipage}
	\caption{A figure}\label{fig:1}
	\end{figure}

\subsection{subcaption in subfigure Umgebung}
	Nam ipsum ligula, eleifend at, accumsan nec, suscipit a,
	ipsum. Morbi blandit ligula feugiat magna. Nunc eleifend consequat
	lorem. Sed lacinia nulla vitae enim. Pellentesque tincidunt purus
	vel magna. Integer non enim. Praesent euismod nunc eu purus. Donec
	bibendum quam in tellus. Nullam cursus pulvinar lectus. Donec et mi.
	Nam vulputate metus eu enim. Vestibulum pellentesque felis eu
	massa.
	
	\begin{figure}[H]
	\centering
	\begin{subfigure}[b]{.48\linewidth}
		\centering
		\includegraphics[width=0.8\linewidth]{images/latex}
		\caption{A subfigure}\label{fig:2a}
	\end{subfigure}%
	\hfill
	\begin{subfigure}[b]{.48\linewidth}
		\centering
		\includegraphics[width=0.8\linewidth]{images/latex}
		\caption{Another subfigure}\label{fig:2b}
	\end{subfigure}
	\caption{A figure}\label{fig:2}
	\end{figure}

% ------------------------------------------------------------
%
\subsection{captionof}

Pellentesque et lectus a est imperdiet egestas. Fusce tempus facilisis lacus. Morbi porttitor eleifend dolor. Integer ante. Integer ornare. Quisque ac urna. Nam egestas, eros sed tempor tincidunt, dolor ante dapibus tellus, eget dapibus metus augue et lectus.

\begin{center}
   \includegraphics[width=0.3\textwidth]{images/latex}
   \captionof{figure}{Ein Beispiel für ein nichtfließendes Bild mit Caption durch `captionof'}
\end{center}

Nullam sed enim quis enim blandit commodo. Nulla ultricies metus eu lorem. Phasellus tincidunt ullamcorper orci. Cras lectus metus, luctus id, vulputate quis, tempor non, nulla. Nullam ut eros et elit faucibus viverra. Cras eu turpis eu urna ullamcorper feugiat. Aliquam bibendum rhoncus tortor. Etiam eu orci. In congue congue quam. Maecenas nec sem quis mi ultrices luctus. Aliquam et risus eu eros molestie pellentesque. Aenean quis purus sit amet lorem porttitor consectetuer.
%

% ============================================================
\section{Tabellen}

Bei allen hier vorgestellten Tabellen sind Befehle in die Tabellen eingefügt 
die den Stil verändern.

% ------------------------------------------------------------
%
\subsection{Tabelle im \emph{booktabs} Stil}
%
\begin{table}[H]
\tablestyle
\begin{tabular}{lll}
\toprule
   Tabellenkopf &
   Tabellenkopf &
   Tabellenkopf \tabularnewline
\midrule
Inhalt & Inhalt  & Inhalt \tabularnewline
Inhalt & Inhalt  & Inhalt \tabularnewline
Inhalt & Inhalt  & Inhalt \tabularnewline
\bottomrule
\end{tabular}
\end{table}
%


% ------------------------------------------------------------
\subsection{Tabelle ohne Kopf}

\begin{table}[H]
\tablestyle
\begin{tabular}{lll}
\hline
Inhalt & Inhalt  & Inhalt \tabularnewline
Inhalt & Inhalt  & Inhalt \tabularnewline
Inhalt & Inhalt  & Inhalt \tabularnewline
\hline
\end{tabular}
\end{table}

%
\subsection{Tabelle mit tabularx}
Tabelle mit einstellbarer Tabellenbreite
%--------------------------------------------------------
\begin{table}[H]
\tablestyle
\begin{tabularx}{\textwidth}{lXXlX}
\toprule
   Tabellenkopf &
   Tabellenkopf &
   Tabellenkopf &
   Tabellenkopf &
   Tabellenkopf \tabularnewline
\midrule
%
   \textit{Beschreibung} & Inhalt & Inhalt & Inhalt & Inhalt \tabularnewline
   \textit{Beschreibung} & Inhalt & Inhalt & Inhalt & Inhalt \tabularnewline
\bottomrule
\end{tabularx}
\caption{Tabelle mit tabularx}
\end{table}
%

% ------------------------------------------------------------
%
\subsection{Tabelle mit alternatierender Farbe}


\begin{table}[H]
   \tablestyle
   \rowcolors{1}{tablerowcolor}{white!100}
   \begin{tabular}{*{2}{v{0.45\textwidth}}}
   \hline
   \rowcolor{tableheadcolor}
Tabellenkopf &
Tabellenkopf \tabularnewline
\hline
%
 Inhalt  & Inhalt \tabularnewline
 Inhalt  & Inhalt \tabularnewline
 Inhalt  & Inhalt \tabularnewline
 Inhalt  & Inhalt \tabularnewline
 Inhalt  & Inhalt \tabularnewline
 Inhalt  & Inhalt \tabularnewline
\hline
   \end{tabular}
\end{table}
%
% ------------------------------------------------------------


% ============================================================
% \section{Wissenschaftliches}

% ============================================================
\section{Mathematik}

\subsection{amsmath Umgebungen}

Im Folgenden sind einige Beispiel für amsmath Umgebungen aufgelistet. Diese 
sind der Dokumentation `mathmode.pdf' von Herbert Voss entnommen.

\subsubsection{align}
\begin{align}
y      & = d\\
y      & = cx + d\\
y_{12} & = bx^{2} + cx + d\\
y(x)   & = ax^{3} + bx^{2}+ cx + d
\end{align}

\subsubsection{alignat}
\begin{alignat}{3}
abc   &= xxx &&= xxxxxxxxxxxx &&= aaaaaaaaa \\
ab    &= yyyyyyyyyyyyyyy &&= yyyy &&= ab
\end{alignat}

\subsubsection{multline}

\begin{multline}
a+b+c+d+e+f\\
+i+j+k+l+m+n
\end{multline}

\subsubsection{cases}
\begin{equation}
x = \begin{cases}
   0 & \text{if A =...}\\
   1 & \text{if B =...}\\
   x & \parbox{5 cm}{%
\flushleft %
this runs with as much text as you like,
but without an automatic linebreak,
it runs out of page ....} %
\end{cases}
\end{equation}
\newpage
\section{Listings}

Listings können in \LaTeX ~mittels des Paketes \textit{listings} dargestellt werden. Das Paket erlaubt Syntax-Highlighting und unterstützt diverse Programmiersprachen sowie unterschiedliche Skriptsprachen. In Listing \ref{lst:bash} ist beispielhaft Java dargestellt.

\begin{lstlisting}[caption=Java example,label=lst:bash]
/**
* This is a doc comment.
*/
package com.ociweb.jnb.lombok;

import java.util.Date;
import lombok.Data;
import lombok.EqualsAndHashCode;
import lombok.NonNull;

$$@Data
$$@EqualsAndHashCode(exclude={"address","city","state","zip"})
public class Person {
enum Gender { Male, Female }

// another comment

%%@NonNull%% private String firstName;
%%@NonNull%% private String lastName;
%%@NonNull%% private final Gender gender;
%%@NonNull%% private final Date dateOfBirth;

private String ssn;
private String address;
private String city;
private String state;
private String zip;
}
\end{lstlisting}

\section{Literatur}
Latex erlaubt die direkte Einbindung und Zitation von Literatur. Dieses kann entweder über die explizite Definition von \textit{bibitems} erfolgen oder die implizite Definition über \textit{BibTex}-Elemente. In diesem Beispiel findet Alternative \textit{A} Anwendung. Egal welche Technik genutzt wird, muss jedoch der IEEE Zitationsstil eingehalten werden (Siehe ``IEEE Citation Style Guide''). Dieser sieht für jeden denkbaren Veröffentlichungstyp ein bestimmtes Format innerhalb der Literaturliste vor. Innerhalb des Textes werden Zitate dann mittels Indexnummer kenntlich gemacht \cite{Chen1993}.

\begin{thebibliography}{1}


\bibitem{North2013}
M.J. North, N.T. Collier, J. Ozik, E. Tatara, M. Altaweel, C.M. Macal, M. Bragen \& P.  Sydelko. ``Complex Adaptive Systems Modeling with Repast Simphony''. In \textit{Journal on Complex Adaptive Systems Modeling}, 1(3), 2013. \href{http://dx.doi.org/10.1186/2194-3206-1-3}{http://dx.doi.org/10.1186/2194-3206-1-3}

\bibitem{Pevere1979}
G. Pevere. ``Infared Nation''. In \textit{The International Journal of Infrared Design}, 33(1), pp. 56-99, Jan. 1979.

\bibitem{Chen1993}
W.K. Chen. \textit{Linear Networks and Systems}. Belmont, CA: Wadsworth, 1993, pp. 123-35.
\end{thebibliography}
% ============================================================


